The Mars rover uses two separate bricks to connect all peripherals. Because of
this the two bricks need to communicate about the status of the different
actuators and sensors. 

This communication could introduce small delays, and as stated in
Section~\ref{sec:requirements} the most imported requirement for the rover is
that is always keeps itself safe. Therefore we propose a layout where the most
important sensors related to safety are connected to the same block as the
two main motors. This ensures that the robot can always keep itself safe,
even when the communication between the two bricks fails.

\begin{table}[H]
	\centering
	\begin{tabular}{|l|l|l|}
		\hline
					& \textbf{Brick 1} & \textbf{Brick 2} \\  \hline
		\hline
		Actuators 	& Left Motor & \\  
					& Right motor & \\  
					& Measurement Motor & \\  
		\hline
		Sensors 	& Light left & Color Sensor \\  
					& Light Right & Gyro Sensor \\  
					& Ultrasonic Front & Touch Sensor Left \\  
					& Ultrasonic Rear & Touch Sensor Right \\  
		\hline
	\end{tabular}
\caption{Connection of the sensors and actuators to the Mars Rover}
\label{tbl:layout}
\end{table}

In this layout Brick 1 is the main control brick, it handles the most important
safety related sensors and all actuators. Brick 2 connects the other sensors and
can then communicate their readings to Brick 1. 

The touch sensor are not considered essential safety sensors as the mars rover
is very sturdy and contact with blocks can already be mostly avoided by the
ultrasonic sensor on the front. 

If it turns out that two ultrasonic sensors on the same brick are problematic
then the front ultrasonic sensor of brick 1 will be interchanged with the
gyro sensor on brick 2. 