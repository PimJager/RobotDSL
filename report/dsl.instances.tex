To complete mission set out in the requirements there are three instances
of our DSL. \texttt{MeasureBlocks} which drives around and measures encountered
blocks, \texttt{PushBlock} which pushes a light block of the table and 
\texttt{ParkInCorner} which drives around, measures the three lakes exactly once
and when all three lakes are found parks the robot in a corner of the table.

\subsubsection{MeasureBlocks}
Attachement 1 shows the implementation of a mission which drives forward and 
measures any encountered blocks. When the robot is in danger of falling of the 
table it will get itself to safety and turn a bit, because of this turning the
robot covers the whole table when driving around.

This mission has six behaviours, \texttt{DetectCliff} and
\texttt{DetectOutsideline} keep the robot from falling of the table. The 
\texttt{MeasureBlockBoth, MeasureBlockRight, MeasureBlockLeft} behaviours 
measure a block when it is detected using the touch sensors, 
\texttt{MeasureBlockLeft} and \texttt{measureBlockRight} will reposition the 
robot so the block is below the arm before measuring. \texttt{DriveForward} is 
the default behaviour, which just makes the robot drive forward in a straight 
line.

\subsubsection{PushBlock}
Attachment 2 shows the implementation of a mission which drives around and 
pushes light blocks of the table. This mission uses the fact that light blocks
do not trigger the touch sensors, while heavy blocks do. Therefore the robot
only tries to detect blocks using the touch sensor. Heavy blocks are then 
avoided and light ones are pushed of the table as a biproduct of driving around.

This mission has four behaviours. \texttt{DetectCliff} and 
\texttt{DetectOutsideLine} behave similar as in the \texttt{MeasureBlocks} 
mission. \texttt{AvoidHeavyBlock} drives away from a heavy block once it has 
detected one using either touch sensor. \texttt{DriveForward} behaves the same
as in the \texttt{MeasureBlocks} mission.

\subsubsection{ParkInCorner}
Attachment 3 shows the implementation of the ParkInCorner mission.
This mission is the most complex of the three missions and describes a 
mission consisting of the parts, first measuring all three lakes exactly once
,while keeping the robot safe from cliffs, driving into lakes or blocks, and 
then parking the robot in a corner of the table. 

To achieve this the mission has 13 behaviours. The mission also uses 4 variables
to track state. For each lake there is a variable which indicated whether it has
been found or not, and there is a variable which indicates if the robot is 
currently tracking a line in search for a corner.

\paragraph{Safety} \texttt{DetectCliff} and \texttt{DetectOutsideLine} work 
similar to the other two missions. \texttt{AvoidBlock} avoids a block when one 
is detected using the front ultrasonic sensor or the either touch sensors.

\paragraph{Detecting lakes} \texttt{MeasurBlueLake, MeasureRedLake} and 
\texttt{MeasureGreenLake} measure a lake when the color sensor is positioned 
over a lake line and the respective lake has not been found (determined using
the respective variable). After measuring the variable which indicates that
that lake is measured is set to \texttt{True}.

\texttt{DetectLakeLineLeft} and \texttt{DetectLakeLineRight} take control when
respectively the left or the right sensor is above a lake line and 
move the robot left or right to try and get the robot in position for a 
successful measurement of the lake. If the color sensor get above the lake line
then the measurement behaviour for that color will take control.

\texttt{DetectLakeColor} takes control when the light sensor is above a lake 
line and none of the measurement behaviours take control. This behaviour 
steers the robot away from the lake if it doesn't need to be measured, to make
sure the robot doesn't drive into the lake.

\paragraph{Parking in the corner} When \texttt{DetectOutsideLine} is active
it checks if all three lakes have been found. If they are then the variable
for tracking line is set to \texttt{True}. This causes \texttt{FollowLine1} to
take control, which will align the robot parallel next to the outside line, 
\texttt{FollowLine2} then takes control and drives the robot to the corner 
very carefully. When any of the safety behaviours take control they set the 
variable for line tracking to false an take the robot to safety. 

\paragraph{Stopping} When the right light sensor detects the outside line 
and the robot is currently tracking the outside line then the mission is 
complete. 

Because the rotate action of the robot is clockwise the outside line followed
will always be on the left hand side of the robot, thus once the right light
sensor detects an outside line a corner is found.
