The Mars Robot DSL and corresponding missions will be developed in an agile way;
the development is split into small sprints. Each sprint consists of designing,
building, integrating and testing a particular function. At the end of each
sprint a working product (DSL and code generation) is delivered. 

These sprints ensure that the development process won't end in some form of
integration hell of all the different sub-parts and shows possible errors
in the design of the robot as early as possible. It also allows to check the
product quick, and often, with the client, which ensures that the developed
product matches the clients expectations. 

\bigskip

The goal of the first sprint will be to write an proof-of-concept rover-program
in Java to test all the different sensors and actuators and the communication
between the two bricks. This has two main benefits, firstly it ensures that the
design of the robot is correct and all the basic function of the robot work.
Secondly, it provides an example for the implementation of the code generation.

Consecutive sprints will consist of implementing the different sensors and
actuators, and writing the mission possible with the sensors and actuators 
implemented so far. 

The selection of the goals for the next sprint is guided by the requirements of 
Section~\ref{sec:requirements}. 

\subsection{Planning}
Table~\ref{tbl:planning} shows the planning for development of the Mars Rover by
listing which requirements from Section~\ref{sec:requirements} will be
implemented in which sprint.
\begin{table}[H]
	\centering
	\begin{tabular}{|l|l|}
		\hline
		Week 	& Goals \\  \hline
		\hline
		Week 1 		& In sprint 1 we will implement a proof of concept to test
						the various sensors and\\   
		(9th Dec)	&	actuators of the mars Rover
						and to test the communication between the two bricks.\\   
		\hline
		Week 2  	& 	The goal of the second sprint is to have a robot which 
						can wonder around and \\  
		(16th Dec)	&	does not fall of the table.\\  
					&	\tabitem F1\\  
					&	\tabitem F11 (only the two driving actuators)\\  
					&	\tabitem F12 (only the rear ultrasonic 
										and light sensors)\\  
					&	\tabitem F13 \\  
					&	\tabitem F14\\  
					&	\tabitem F17\\  
					&	\tabitem U2\\  
					&	\tabitem R1\\  
		\hline
		Week 3		& The goal of the third sprint is to implement the detection
						, measuring and pushing \\  
		(6th Jan)	& of rocks and lakes. \\  
					& 	\tabitem F2\\  
					& 	\tabitem F3\\  
					& 	\tabitem F4\\  
					& 	\tabitem F5\\  
					& 	\tabitem F8\\  
					&  	\tabitem F9\\  
					& 	\tabitem F11 (the measurement actuator)\\  
					& 	\tabitem F12\\  
		\hline
		Week 4 		& The goal of this sprint is to implement the actual 
						missions\\  
		(13th Jan) 	& 	\tabitem F18\\  
					& 	\tabitem P1\\  
					&	\tabitem P2\\  
		\hline
	\end{tabular}
\caption{Planning for development of the mars rover}
\label{tbl:planning}
\end{table}
If there is time left after implementing all the requirements with priority 
`must' then the following requirements with priority `should' shall be 
implemented first:
\begin{itemize}
	\item F21 
	\item F22
	\item U2
\end{itemize}