The Mars Robot DSL and corresponding missions will be developed in an agile way;
the development is split into small sprints. Each sprint consists of designing,
building, integrating and testing a particular function. At the end of each
sprint a working product (DSL and code generation) is delivered. 

These sprints ensure that the development process won't end in some form of
integration hell of all the different sub-parts and shows possible errors
in the design of the robot as early as possible. It also allows to check the
product quick, and often, with the client, which ensures that the developed
product matches the clients expectations. 

\bigskip

The goal of the first sprint will be to write an proof-of-concept rover-program
in Java to test all the different sensors and actuators and the communication
between the two bricks. This has two main benefits, firstly it ensures that the
design of the robot is correct and all the basic function of the robot work.
Secondly, it provides an example for the implementation of the code generation.

Consecutive sprints will consist of implementing the different sensors and
actuators, and writing the mission possible with the sensors and actuators 
implemented so far. 

The selection of the goals for the next sprint is guided by the requirements of 
Section~\ref{sec:requirements}. 