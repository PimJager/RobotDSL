The program a mars rover using our DSL two Java files are generated; one for
each brick on the rover. 

\subsubsection{Right Brick} 
The Java file for the right brick is a static file containing a Java class 
which controls the right brick. As per Section~\ref{sec:layout} the 
right brick contains only sensors and actuators. Therefore the sole
task of the right brick is to continuously read its sensor values and send them
to the left brick via Bluetooth. 

\subsubsection{Left Brick}
The Java file for the left brick contains a mixture of generated and static 
code. It contains a main class and classes for each specified behaviour. 
Attachment 4 shows this file for the MeasureBlocks mission, with each generated
line prefixed with a $>$. Below we will describe the general layout of the code
in a pseudo Java syntax.

\begin{lstlisting}
< list of necesarry imports >

public class Robot {

	< variables needed for actuators and sensors >

	< storage of sensor values send via Bluetooth from right brick >

	< variables needed for Bluetooth connection >

	< GENERATED list of constants eg. >
	public static int CONSTANT_NAME = (40);

	//is the robot running? 
	public static int _running = 1;

	< GENERATED list of variables eg. >
	public static int variableName;

	public static void main(..) {
		init();
		Behavior[] bList = {
							< GENERATED list of behaviours eg. >
							new BehaviourName(),
							< Every robot has this behaviour for 'stops when:' >
							,new DefaultQuitBehaviour() 
							};
		Arbitrator ar = new Arbitrator(bList); ar.start();
	}

	< init() function: sets up Bluetooth, resets sensors and 
			actuators, starts the bluetooth and lcd threads. >

	< updateSensors() update the sensors attached to this brick >

	< normalisation functions to allow typeless variables in the DSL
			in the typed JAVA language >

	< Thread which prints the current sensor values and active 
			behaviour to the LCD >

	< Thread wich reads incoming bluetooth messages from the right 
			brick and saves the sensor values >

	< function for the measurement action >

	< GENERATED list of subroutines eg. >
	public static function routineName(){
		Robot.rightMotor.stop();
		variableName = CONSTANT_NAME;
	}
}

class DefaultQuitBehaviour implements Behavior {	
	@Override
	public boolean takeControl() {
		return < GENERATED 'stops when' expression >;
	}
	< action() and suppress() functions >
}

< GENERATED Behaviour classes which have the following form >
class < GENERATED name of behaviour > implements Behaviour {
	private boolean _supressed = true;
	public boolean takeControl() {
		Robot.updateSensors();
		return Robot.makeBool(Robot._running) &&
				< GENERATED 'takes control when' expression>;
	}
	public void action() {
		_supressed = false;
		Robot.updateSensors();
		< GENERATED list of expressions for this behaviours 'Does:' 
		  with a supression check before every expression >
		if(_supressed) return;
		Robot.routineName;
	}
	public void suppress() {_supressed = true;}
}
\end{lstlisting}

Note that when generating expressions care has to be taken that all expressions
have the correct type. Therefore the robot contains some normalisation 
functions which convert from bool to int and from int to bool. 