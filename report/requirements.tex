The Mars Robot shall adhere to the following requirements, which are set up
using the FURPS model, and are prioritized using the MOSCOW method.

\setcounter{TCC}{1}
\begin{table}[H]
	\centering
	\begin{tabular}{|l|l|l|}
		\hline
		\multicolumn{3}{|c|}{Functional requirements} \\  \hline
		\hline
		Identifier & Priority & Description \\  
		\hline
		\hline
		F\doTCC  & Must & The robot must not fall of the table \\  \hline
		F\doTCC  & Must & The robot must be able to detect colors \\  \hline
		F\doTCC  & Must & The robot must be able to detect rocks \\  \hline
		F\doTCC  & Must & The robot must be able to ``measure'' rocks \\  \hline
		F\doTCC  & Must & The robot must be able to push light-weight rocks 
			\\  \hline
		F\doTCC  & Should & The robot should be able to avoid lakes \\  \hline
		F\doTCC  & Could & The robot could be able to park in a corner 
			\\  \hline
		F\doTCC  & Must & The robot must be able to ``measure'' lakes exactly 
			once \\  \hline
		F\doTCC & Must & The robot must be able to avoid rocks \\  \hline
		F\doTCC & Could & The generated programs could work on both robots
			\\  \hline
		F\doTCC & Must & The DSL must allow control of the three actuators 
			\\  \hline
		F\doTCC & Must & The DSL must provide readouts of the three sensors
			\\  \hline
		F\doTCC & Must & The DSL must provide a method of controlling actuators
			based on the values of sensors \\  \hline
		F\doTCC & Must & The DSL must provide a method of prioritizing different
			actions \\  \hline
		F\doTCC & Should & The DSL should abstract over commonly used 
			functionality \\  \hline
		F\doTCC & Could & The DSL could provide a method to specify and execute
			user specified subroutines \\  \hline
		F\doTCC & Must & The DSL must generate a valid Java program which can be
			run on the mars rover. \\  
		\hline
	\end{tabular}
\caption{Functional requirements of Mars Rover}
\label{tbl:functionalReq}
\end{table}

\setcounter{TCC}{1}
\begin{table}[H]
	\centering
	\begin{tabular}{|l|l|l|}
		\hline
		\multicolumn{3}{|c|}{Usability requirements} \\  \hline
		\hline
		Identifier & Priority & Description \\  
		\hline
		\hline
		U\doTCC & Should & The mapping between a generated Java program and a
			specification in the DSL should be easy to understand \\  \hline
		U\doTCC & Must & The DSL must be easy to read and understand \\  \hline
		U\doTCC & Would & The robot prints a trace of behaviors and values on
			the LCD for easy debugging \\  
		\hline
	\end{tabular}
\caption{Usability requirements of Mars Rover}
\label{tbl:usabilityReq}
\end{table}

\setcounter{TCC}{1}
\begin{table}[H]
	\centering
	\begin{tabular}{|l|l|l|}
		\hline
		\multicolumn{3}{|c|}{Reliability requirements} \\  \hline
		\hline
		Identifier & Priority & Description \\  
		\hline
		\hline
		R1 & M & Sample description \\  
		\hline
	\end{tabular}
\caption{Reliability requirements of Mars Rover}
\label{tbl:reliabilityReq}
\end{table}

\begin{table}[H]
	\centering
	\begin{tabular}{|l|l|l|}
		\hline
		\multicolumn{3}{|c|}{Performance requirements} \\  \hline
		\hline
		Identifier & Priority & Description \\  
		\hline
		\hline
		P1 & M & Sample description \\  
		\hline
	\end{tabular}
\caption{Performance requirements of Mars Rover}
\label{tbl:performanceReq}
\end{table}

\begin{table}[H]
	\centering
	\begin{tabular}{|l|l|l|}
		\hline
		\multicolumn{3}{|c|}{Supportability requirements} \\  \hline
		\hline
		Identifier & Priority & Description \\  
		\hline
		\hline
		S1 & M & Sample description \\  
		\hline
	\end{tabular}
\caption{Supportability requirements of Mars Rover}
\label{tbl:supportabilityReq}
\end{table}